
\section*{Code for Random Offspring ( 1+ Poisson(1)) Distribution and Random Mating Function}
\begin{lstlisting}[language=Python]
import numpy as np
import random
import matplotlib.pyplot as plt
\end{lstlisting}
\begin{lstlisting}[language=Python]
class Person:
    def __init__(self, gender):
        self.gender = gender
        self.mother = None
        self.father = None
        self.generation = None

def create_first_generation():
    M = Person("M")
    F = Person("F")
    M.mother = F.father = None
    M.father = F.mother = None
    M.generation = F.generation = 1
    return [M, F]

def mate_random(parent1, parent2, generation):
    num_male_children = 1+ np.random.poisson(1)  #mean number of male children is 1
    num_female_children = 1+ np.random.poisson(1)  # mean number of female children is 1
    #print(num_male_children, num_female_children)

    children = []
    for _ in range(num_male_children):
        male_child = Person("M")
        male_child.mother = parent1
        male_child.father = parent2
        male_child.generation = generation
        children.append(male_child)

    for _ in range(num_female_children):
        female_child = Person("F")
        female_child.mother = parent1
        female_child.father = parent2
        female_child.generation = generation
        children.append(female_child)

    return children

def create_next_generation_random(current_generation, generation_number):
    next_generation = []
    # Track individuals who have already mated
    already_mated = []

    # Shuffle the order of individuals in the current generation
    random.shuffle(current_generation)

    for parent1 in current_generation:
        if parent1.gender == "F" and parent1 not in already_mated:

            # Find a random male who hasn't mated yet
            for parent2 in current_generation:
                if parent2.gender == "M" and parent2 not in already_mated:
                    children = mate_random(parent1, parent2, generation_number)
                    next_generation.extend(children)
                    # Mark both parents as already mated
                    already_mated.append(parent1)
                    already_mated.append(parent2)
                    break  # Move to the next female once a male is found
    return next_generation
\end{lstlisting}
\begin{lstlisting}[language=Python]


\end{lstlisting}
\begin{lstlisting}[language=Python]
def find_common_ancestor(person1, person2):
    def find_all_ancestors(person):
        ancestors = []
        current_generation = [person]

        while current_generation:
            next_generation = []
            for ancestor in current_generation:
                if ancestor and ancestor not in ancestors:  # Avoid duplicates
                    ancestors.append(ancestor)
                    next_generation.extend([ancestor.mother, ancestor.father])
            current_generation = next_generation

        return ancestors

    # Find all ancestors for both individuals
    #[person1's mother, person1's father, grandmother1, grandfather1, great-grandmother1, great-grandfather1]
    ancestors1 = find_all_ancestors(person1)
    ancestors2 = find_all_ancestors(person2)

    # Find common ancestors
    #ancestors1 = [grandmother1, grandfather1, great-grandmother1, great-grandfather1]
    #ancestors2 = [grandmother1, grandfather2, great-grandmother1]
    #common_ancestors = [grandmother1, great-grandmother1]
    common_ancestors = [ancestor for ancestor in ancestors1 if ancestor in ancestors2]

    # If there are common ancestors, find the earliest (lowest generation number)
    if common_ancestors:
        #common_ancestors = [grandmother1 (gen=3), great-grandmother1 (gen=2)] => answer should be 3
        earliest_common_ancestor = max(common_ancestors, key=lambda x: x.generation)
        return earliest_common_ancestor.generation  # Return generation of the earliest common ancestor
    else:
        return None  # No common ancestor found

\end{lstlisting}
\begin{lstlisting}[language=Python]
def find_mfcommon_ancestor(person1, person2):
    def find_ancestors(person, gender):
        ancestors = []
        while person:
            if gender == "F":
                person = person.mother
            else:
                person = person.father
            if person:
                ancestors.append(person)
        return ancestors

    female_ancestors1 = find_ancestors(person1, "F")
    female_ancestors2 = find_ancestors(person2, "F")
    male_ancestors1 = find_ancestors(person1, "M")
    male_ancestors2 = find_ancestors(person2, "M")

    common_female_ancestor = None
    for i in range(min(len(female_ancestors1), len(female_ancestors2))):
        if female_ancestors1[i] == female_ancestors2[i]:
            common_female_ancestor = female_ancestors1[i]
            break

    common_male_ancestor = None
    for ancestor in male_ancestors1:
        if ancestor in male_ancestors2:
            common_male_ancestor = ancestor
            break

    return common_female_ancestor, common_male_ancestor
\end{lstlisting}
\begin{lstlisting}[language=Python]

\end{lstlisting}
\begin{lstlisting}[language=Python]
common_ancestorgenall = []
male_ancestorgenall = []
female_ancestorgenall = []
for k in range(10000):
    if k %10 == 0:
        print(k)

    common_ancestor_generations = []
    generations = []
    current_generation = create_first_generation()
    generations.append(current_generation)

    common_female_generations = []
    common_male_generations = []


    for i in range(2, 11):
        next_generation = create_next_generation_random(current_generation, i)
        generations.append(next_generation)
        current_generation = next_generation

    for _ in range(10000):
        # Select two random individuals from the last generation
        last_generation = generations[-1]
        person1, person2 = random.sample(last_generation, 2)

        # Find the generation of the earliest common ancestor
        common_ancestor_generation = find_common_ancestor(person1, person2)

        if common_ancestor_generation is not None:
            common_ancestor_generations.append(common_ancestor_generation)

        common_female_ancestor, common_male_ancestor = find_mfcommon_ancestor(person1, person2)

        if common_female_ancestor is not None:
            common_female_generations.append(common_female_ancestor.generation)

        # Store the generation number for common male ancestor
        if common_male_ancestor is not None:
            common_male_generations.append(common_male_ancestor.generation)


    common_ancestorgenall.append(common_ancestor_generations)
    male_ancestorgenall.append(common_male_generations)
    female_ancestorgenall.append(common_female_generations)



\end{lstlisting}
\begin{lstlisting}[language=Python]
np.unique(common_ancestorgenall, return_counts = True)
\end{lstlisting}
\begin{lstlisting}[language=Python]
np.unique(male_ancestorgenall, return_counts = True)
\end{lstlisting}
\begin{lstlisting}[language=Python]
np.unique(female_ancestorgenall, return_counts = True)
\end{lstlisting}


\section*{Code for Deterministic Offspring (2 Male, 2 Female) Distribution and Random Mating Function}
\begin{lstlisting}[language=Python]
import numpy as np
import random
import matplotlib.pyplot as plt
\end{lstlisting}
\begin{lstlisting}[language=Python]
class Person:
    def __init__(self, gender):
        self.gender = gender
        self.mother = None
        self.father = None
        self.generation = None

def create_first_generation():
    M = Person("M")
    F = Person("F")
    M.mother = F.father = None
    M.father = F.mother = None
    M.generation = F.generation = 1
    return [M, F]

def mate(parent1, parent2, generation):
    child1 = Person("M")
    child2 = Person("F")
    child3 = Person("M")
    child4 = Person("F")
    child1.mother = child2.mother = child3.mother = child4.mother = parent1
    child1.father = child2.father = child3.father = child4.father= parent2
    child1.generation = child2.generation = child3.generation = child4.generation = generation
    return child1, child2, child3, child4

def create_next_generation(current_generation, generation_number):
    next_generation = []
    # Track individuals who have already mated
    already_mated = []

    # Shuffle the order of individuals in the current generation
    random.shuffle(current_generation)

    for parent1 in current_generation:
        if parent1.gender == "F" and parent1 not in already_mated:

            # Find a random male who hasn't mated yet
            for parent2 in current_generation:
                if parent2.gender == "M" and parent2 not in already_mated:
                    child1, child2, child3, child4 = mate(parent1, parent2, generation_number)
                    next_generation.extend([child1, child2, child3, child4])
                    # Mark both parents as already mated
                    already_mated.append(parent1)
                    already_mated.append(parent2)
                    break  # Move to the next female once a male is found
    return next_generation


\end{lstlisting}
\begin{lstlisting}[language=Python]
def find_common_ancestor(person1, person2):
    def find_all_ancestors(person):
        ancestors = []
        current_generation = [person]

        while current_generation:
            next_generation = []
            for ancestor in current_generation:
                if ancestor and ancestor not in ancestors:  # Avoid duplicates
                    ancestors.append(ancestor)
                    next_generation.extend([ancestor.mother, ancestor.father])
            current_generation = next_generation

        return ancestors

    # Find all ancestors for both individuals
    #[person1's mother, person1's father, grandmother1, grandfather1, great-grandmother1, great-grandfather1]
    ancestors1 = find_all_ancestors(person1)
    ancestors2 = find_all_ancestors(person2)

    # Find common ancestors
    #ancestors1 = [grandmother1, grandfather1, great-grandmother1, great-grandfather1]
    #ancestors2 = [grandmother1, grandfather2, great-grandmother1]
    #common_ancestors = [grandmother1, great-grandmother1]
    common_ancestors = [ancestor for ancestor in ancestors1 if ancestor in ancestors2]

    # If there are common ancestors, find the earliest (lowest generation number)
    if common_ancestors:
        #common_ancestors = [grandmother1 (gen=3), great-grandmother1 (gen=2)] => answer should be 3
        earliest_common_ancestor = max(common_ancestors, key=lambda x: x.generation)
        return earliest_common_ancestor.generation  # Return generation of the earliest common ancestor
    else:
        return None  # No common ancestor found

\end{lstlisting}
\begin{lstlisting}[language=Python]
def find_mfcommon_ancestor(person1, person2):
    def find_ancestors(person, gender):
        ancestors = []
        while person:
            if gender == "F":
                person = person.mother
            else:
                person = person.father
            if person:
                ancestors.append(person)
        return ancestors

    female_ancestors1 = find_ancestors(person1, "F")
    female_ancestors2 = find_ancestors(person2, "F")
    male_ancestors1 = find_ancestors(person1, "M")
    male_ancestors2 = find_ancestors(person2, "M")

    common_female_ancestor = None
    for i in range(min(len(female_ancestors1), len(female_ancestors2))):
        if female_ancestors1[i] == female_ancestors2[i]:
            common_female_ancestor = female_ancestors1[i]
            break

    common_male_ancestor = None
    for ancestor in male_ancestors1:
        if ancestor in male_ancestors2:
            common_male_ancestor = ancestor
            break

    return common_female_ancestor, common_male_ancestor
\end{lstlisting}
\begin{lstlisting}[language=Python]

\end{lstlisting}
\begin{lstlisting}[language=Python]
common_ancestorgenall = []
male_ancestorgenall = []
female_ancestorgenall = []
for k in range(1000):
    if k %10 == 0:
        print(k)

    common_ancestor_generations = []
    generations = []
    current_generation = create_first_generation()
    generations.append(current_generation)

    common_female_generations = []
    common_male_generations = []


    for i in range(2, 11):
        next_generation = create_next_generation(current_generation, i)
        generations.append(next_generation)
        current_generation = next_generation
    for _ in range(1000):
        # Select two random individuals from the last generation
        last_generation = generations[-1]
        person1, person2 = random.sample(last_generation, 2)

        # Find the generation of the earliest common ancestor
        common_ancestor_generation = find_common_ancestor(person1, person2)

        if common_ancestor_generation is not None:
            common_ancestor_generations.append(common_ancestor_generation)

        common_female_ancestor, common_male_ancestor = find_mfcommon_ancestor(person1, person2)

        if common_female_ancestor is not None:
            common_female_generations.append(common_female_ancestor.generation)

        # Store the generation number for common male ancestor
        if common_male_ancestor is not None:
            common_male_generations.append(common_male_ancestor.generation)
            
        common_ancestorgenall.append(common_ancestor_generation)
        male_ancestorgenall.append(common_male_ancestor)
        female_ancestorgenall.append(common_female_ancestor)



\end{lstlisting}
\begin{lstlisting}[language=Python]
np.unique(common_ancestorgenall, return_counts = True)
\end{lstlisting}
\begin{lstlisting}[language=Python]
np.unique(male_ancestorgenall, return_counts = True)
\end{lstlisting}
\begin{lstlisting}[language=Python]
np.unique(female_ancestorgenall, return_counts = True)
\end{lstlisting}
